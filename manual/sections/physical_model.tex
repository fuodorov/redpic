\section{Physical model}\label{sec:physical-model}

First, we write the main quantities in the difference scheme, then we consider the algorithm.

\subsection{Basic values}\label{subsec:basic-values}
At the initial time $t$ set Cartesian coordinates of the $i-$particles $\overrightarrow{r_{i}} = (x_i, y_i, z_i)$ and the initial pulses $\overrightarrow{p_{i}} = (p_{x_i}, p_{y_i}, p_{z_i})$.
Minimum time step is set $\delta t$.

For convenience, made dimensionless physical quantities (with tilde) (\ref{eq:r_tilde} -~\ref{eq:H_tilde}):
\begin{equation}
    \label{eq:r_tilde}
    \widetilde{\overrightarrow{r_i}} = \frac{\overrightarrow{r_i}}{c \delta t},
\end{equation}
\begin{equation}
\label{eq:E_tilde}
    \widetilde{\overrightarrow{E_i}} = \frac{\overrightarrow{E_i}e}{mc} \delta t,
\end{equation}
\begin{equation}
\label{eq:H_tilde}
    \widetilde{\overrightarrow{H_i}} = \frac{\overrightarrow{H_i}e}{mc} \delta t,
\end{equation}
where $m$ is the speed of light, $e$ is the particle charge and $m$ is the particle mass.

\subsection{Algorithm}\label{subsec:algorithm}

Consider the relativistic difference scheme algorithm step by step~\cite{Logatchev2011}:

\begin{enumerate}
    \label{en:algorithm}
    \item Calculation of electric fields $\overrightarrow{E_i}$ at the points where the particles are.
    \item Pulse increment:
    \begin{equation}
        \overrightarrow{p_i} = \overrightarrow{p_i} + 2\cdot\overrightarrow{E_i},
    \end{equation}
    where the coefficient means that the increment of the pulse is made for the entire time interval $2\delta t.$
    \item New speeds are calculated by new impulses:
    \begin{equation}
        \overrightarrow{v_i} = \frac{\overrightarrow{p_i}}{\gamma_i},
    \end{equation}
    where $\gamma_i = \sqrt{1 + \overrightarrow{p_i}^2}.$
    \item At new speeds, new coordinates are calculated:
    \begin{equation}
        \overrightarrow{r_i} = \overrightarrow{r_i} + 1\cdot\overrightarrow{v_i},
    \end{equation}
    where the coefficient means that the increment of the coordinate is made for the time interval $\delta t.$
    \item Calculation of magnetic fields $\overrightarrow{H_i}$ at the new points where the particles are.
    \item Calculated velocity values (after rotation in a magnetic field):
    
    \[
            b_1 = 1 - \frac{H_i^2}{\gamma_i},
            b_2 = 1 + \frac{H_i^2}{\gamma_i}, 
            b_3 = 2\cdot \frac{\overrightarrow{v_i}\cdot\overrightarrow{H_i}}{\gamma_i},
    \]
    
    \[
    \overrightarrow{f_i} = 2\cdot \frac{\overrightarrow{v_i}\times \overrightarrow{H_i}}{\gamma_i},
    \]
    
    \begin{equation}
        \overrightarrow{v_i} = \frac{\overrightarrow{v_i}b_1 + \overrightarrow{f_i} + \frac{\overrightarrow{H_i}}{\gamma_i}b_3}{b_2}.
    \end{equation}
    \item At new speeds, new coordinates are calculated:
    \begin{equation}
        \overrightarrow{r_i} = \overrightarrow{r_i} + 1\cdot\overrightarrow{v_i},
    \end{equation}
    where the coefficient means that the increment of the coordinate is made for the time interval $\delta t.$   
    \item New impulses are calculated according to the new rates:
    \begin{equation}
        \overrightarrow{p_i} = \overrightarrow{v_i}\gamma_i.
    \end{equation}
    
    
\end{enumerate}

One cycle is performed in a time interval $2\delta t.$
